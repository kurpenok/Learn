\input{preamble.tex}

\begin{document}
    
    Структуризация документа:
    \section{Раздел}
    \subsection{Подраздел}
    \dots % Двух уровней вложенности хватает
    \subsection*{Подраздел без номера}
    \dots % Такие команды можно использовать как подзаголовки
    \subsection{Подраздел с номером}
    \dots % Нумерация не нарушается

    Стиль текста:
    \textbf{Жирный}, \textit{курсивный}, обычный.
    {\bfseries \itshape Здесь выделено}, а здесь нет.
    
    Размер текста:
    {\Large Большой текст}, текст поменьше, {\small вообще маленький текст, жестб...}.

     Выравнивание:
     \begin{center}
         Центрированный текст.
     \end{center}

    \begin{flushright}
        Текст, выровненный по правому краю.
    \end{flushright}
    
    % Пакет окружения для теорем
    \usepackage{amsthm}
    \theoremstyle{plain}
    \newtheorem{theorem}{Теорема}
    \newtheorem{lemma}{Лемма}

    \begin{lemma}
        1 + 1 = 3
    \end{lemma}

    \begin{theorem}
        2 + 2 = 5
    \end{theorem}

\end{document}

